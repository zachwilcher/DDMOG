\documentclass{article}

\usepackage{mathtools}
\usepackage{amsmath}
\usepackage{amssymb}
\usepackage{amsthm}

\theoremstyle{plain}
\newtheorem{theorem}{Theorem}[section]
\newtheorem{lemma}[theorem]{Lemma}
\newtheorem{proposition}[theorem]{Proposition}
\newtheorem{corollary}{Corollary}[section]

\theoremstyle{definition}
\newtheorem{definition}{Definition}[section]
\newtheorem{conjecture}{Conjecture}[section]
\newtheorem{example}{Example}[section]
\newtheorem{question}{Question}[section]
\newtheorem{remark}{Remark}[section]

\renewcommand{\Im}{\mathrm{Im}\,}
\newcommand{\Ker}{\mathrm{Ker}\,}

\begin{document}
A review of some concepts from Linear Algebra and facts about Abelian groups.

\section{Exact Sequences}
\begin{definition}
    An exact sequence of groups \(\{H_k\}\) is one where
    there exists a homomorphism \(h_k: H_{k-1} \rightarrow H_k\) for each \(k\)
    such that \(\Im(h_{k+1}) = \Ker(h_k)\) for each \(k\).
\end{definition}

\begin{remark}
    If the following sequence is exact, then \(g(f(A))\) is trivial.
    \[
        0 \rightarrow A \xrightarrow{f} B \xrightarrow{g} C \rightarrow 0
    \]
\end{remark}
\begin{proof}
Let \(x \in A\), then \(f(x) \in \Ker(g)\) so \(g(f(x)) = 0\).
\end{proof}

\begin{lemma}
    If the following sequence is exact, then \(C \cong B / \Im(f)\).
    \[
        0 \rightarrow A \xrightarrow{f} B \xrightarrow{g} C \rightarrow 0
    \]
\end{lemma}
\begin{proof}
By the fundamental theorem of group homomorphisms, \(C \cong B / \Ker(g)\).
By exactness, \(C \cong B / \Im(f)\).
\end{proof}

\end{document}