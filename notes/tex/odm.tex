\section{Definitions}
\begin{definition}
    Given an abelian group \(\Gamma\),
    a directed graph \(G\) is \(\Gamma\)-orientable distance magic (\(\Gamma\)-ODM) if and only if
    there exists a bijection \(\ell: V(G) \rightarrow \Gamma\) such that
    for some \(k \in \Gamma\) and for every vertex \(x \in V(G)\),
    \[
    k = \sum_{y \in N^+(G,x)} \ell(y)
    - \sum_{y \in N^-(G,x)} \ell(y)
    \]
    where \(N^+(G,x)\) and \(N^-(G,x)\) are the in-neighborhood and out-neighborhood of \(x\) respectively.
    We call \(\ell\) a \(\Gamma\)-ODM labeling of \(G\) and \(k\) the magic constant of \(\ell\).
\end{definition}