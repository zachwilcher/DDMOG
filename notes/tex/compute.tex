\section{Computation}

\begin{proposition}
There are 
\(\displaystyle
\sum_{k=0}^{\binom{n}{2}} \binom{\binom{n}{2}}{k} 2^k = 3^{\binom{n}{2}}
\)
oriented graphs on \(n\) vertices.
\end{proposition}
\begin{proof}
There are \(\binom{n}{2}\) edges in \(K_n\).
For each number \(k\) of possible edges from \(0\) to \(\binom{n}{2}\)
we have \(\binom{n(n+1)/2}{k}\) ways to put undirected edges between
vertices.
For each placement of \(k\) possible edges, there
are \(2^k\) possible orientations of those edges.
Using the binomial theorem, the resulting sum can be simplified.
\end{proof}

So, an algorithm that checks if every possible oriented graph on \(n\) vertices is DDM is at least \(O(3^{\binom{n}{2}})\).
If 50000 graphs can be checked in 1 second, then to check every graph on 7 vertices
we would need at least 58 hours.

To more efficiently find all DDMOGs, we can reduce the search space by looking more closely at the requirement
for a labeled oriented graph to be DDM.
Suppose \(\vec{G}\) is an order \(n\) oriented graph with a labeling \(\ell: V(\vec{G}) \rightarrow \{1, 2, \cdots, n\}\)
and adjacency matrix \(A = (a_{i,j})\).
We use the convention that \(a_{i,j} = 1\) if \(v_i\) is directed at \(v_j\) and \(a_{i,j} = 0\) otherwise.
For any vertex \(v_i\), we then have the following equation.
\begin{equation}
    \label{eqn:weight_equation}
    \wt(v_i) = \sum_{j=1}^n (a_{j,i}-a_{i,j}) \ell(v_j)
\end{equation}
All \(n\) equations then can be expressed as
\begin{equation}
    \label{eqn:matrix_weight_equation}
\vec{w} = (A^{T} - A)\vec{x}
\end{equation}
where \(\vec{x}\) is the label column vector \([\ell(v_1), \ell(v_2), \cdots, \ell(v_n)]^{T}\)
and \(\vec{w}\) is the weight vector \([\wt(v_1), \wt(v_2), \cdots, \wt(v_n)]^T\).

Let \(S = A^{T} - A\) and \(\vec{s}_i\) be the \(i\)th column of \(S\).
Since \(S\) is symmetric, Equation \ref{eqn:weight_equation} can then be expressed concisely
as \(\vec{s}_i \cdot \vec{x} = \wt(v_i)\).
This sort of notation is used extensively in \cite{aceska2026newresults}.

We make some remarks
\begin{enumerate}
\item If \((\vec{G}, \ell)\) is DDM, then \(\wt(v_i) = 0\) (see Theorem 1 in \cite{altman2024difference}),
Equation \ref{eqn:weight_equation} is just \(\vec{s_i} \cdot \vec{x} = 0\).

\item The components of \(\vec{s}_i\) belong to \(\{-1, 0, 1\}\).

\item By identifying \(V(\vec{G}) = \{v_1, \cdots, v_n\}\) such that \(\ell(v_i) = i\), the label vector
is \(\vec{x} = [1, 2, \cdots, n]^T\).
\end{enumerate}

With this information, we can rephrase Equation \ref{eqn:weight_equation} once more as
\begin{equation}
    \label{eqn:skew_weight_equation}
0 = \sum_{j=1}^n j \cdot s_{i, j}
\end{equation}
So, if we wish to know what the possible ways are to add oriented edges incident to \(v_i\)
we need to find a collection of \(n\) numbers \(\{s_{i,j}\}_{j=1}^n\) each chosen from
\(\{-1, 0, 1\}\) such that \(s_{i, i} = 0\) and Equation \ref{eqn:skew_weight_equation} holds.

This is a variation of the famous subset sum problem that asks is there a way to
select a subset of numbers from a given set such that their sum is equal to some target number.
In our case, we want to select numbers from the set \(\{1, 2, \cdots, n\}\)
such that the sum is zero. However, we also consider the possibility of selecting the negative of each number from our set.
The fastest known algorithms for solving the subset sum problem run in exponential time.
Although, this is much faster than checking every possible oriented graph on \(n\) vertices.


