\section{Sparsity}

\begin{proposition}
    A DDMOG of order \(n\) must have at least \(\ceil{\frac{3n}{2}}\) edges.
\end{proposition}
\begin{proof}
    Property 1 shown in \cite{altman2024difference} states that each vertex in a graph
    with a DDM labeling must have degree at least \(3\).
    So, if \(G\) has a DDM labeling, then
    \[
    2\abs{E(G)} = \sum_{v \in V(G)} \deg(v) \ge \sum_{v \in V(G)} 3 = 3n.
    \]
    Since the number of edges must be an integer, it follows that
    \(\abs{E(G)} \ge \ceil{\frac{3n}{2}}\).
\end{proof}


\begin{definition}
    \(H\) is the order \(10\) graph
    that is two triangles with their vertices attached to the fingertips of \(K_{1,3}\).
\end{definition}
Some further properties of \(H\) are available here: 

\url{https://houseofgraphs.org/graphs/32533}.


\begin{definition}
    \(O\) is the order \(13\) graph
    that looks kind of like \(H\) but with an additional copy of \(K_{1,3}\)
    that attaches in a similar way.
\end{definition}

\begin{definition}
    \(W_k\) is the wheel on \(k\) vertices.
\end{definition}

\begin{conjecture}
    Let \(x\) be a non-negative integer,
    then the following families of graphs are DDMO.
    Furthermore, each graph family has minimal size.

\begin{tabular}{c | c }
\(n \pmod{12}\) & Family \\
\hline
0 & \(2(x + 1) K_{3,3}\) \\
1 & \(xK_{3,3}\cup O\) \\
2 & N/A \\
3 & \(x K_{3,3} \cup H \cup W_4\) \\
4 & \((x+1) K_{3,3} \cup H\) \\
5 & \(x K_{3,3} \cup W_4\) \\
6 & N/A \\
7 & \((x+1) K_{3,3}\cup O\) \\
8 & \((x+1) K_{3,3} \cup 2H\) \\
9 & \((x+1) K_{3,3} \cup H \cup W_4\) \\
10 & N/A \\
11 & \((x+1) K_{3,3} \cup W_4\)
\end{tabular}
\end{conjecture}

\begin{conjecture}
    There are no DDMOGs of order \(n\)
    with minimal size when \(n \equiv 2 \pmod{4}\).
\end{conjecture}

\begin{corollary}
    If \(n \ge 10\) and \(n \not \equiv 2 \pmod{4}\),
    then there exists a DDMOG of order \(n\) with minimal size.
\end{corollary}
\begin{proof}
    \(n = 20\) is the only exception to the conjectures above.
    But, there exists a connected DDMOG of order \(20\) with minimal size.
\end{proof}



\begin{proposition}
Let \(A\) be an \(n/3\) by \(n\) matrix where row \(i\)
consists of \(3 * (i - 1)\) zeros, followed by \((1, -1, -1)^T\), following by
\(3 * (n - i)\) more zeros.
If a permutation of the label vector \((1,2, \cdots, n)^T\) is in the kernel
of \(A\), then \(n \equiv 0 \pmod{4}\) or \(n \equiv 3 \pmod{4}\).
\end{proposition}

\begin{proof}
Note that \(n\) must be a multiple of \(3\) for the matrix multiplication
to make sense.
Let \(x\) be the label vector \((1,2,\cdots, n)^T\) and \(P\) be a permutation matrix.
Let \(u = Px\).
If \(Au = 0\), then
\[
u_{i + 1} - u_{i+2}  - u_{i+3} = 0 \Leftrightarrow u_{i+1} = u_{i+2} + u_{i+3}
\]
for any \(i \equiv 0 \pmod{3}\).
Observe:
\[
\sum_{i=1}^n u_i = \sum_{i=1}^n i = \frac{n(n+1)}{2}
\]
However,
\[
\sum_{i=1}^n u_i = \sum_{i=0}^{n/3} \left(u_{3i + 1} + u_{3i + 2} + u_{3i + 3}\right) = 2 \sum_{i=0}^{n/3} u_{3i + 1}
\]
So, \(n(n+1)\) is a multiple of \(4\).
Suppose \(n = 4q + r\) for \(0 \le r < 4\).
\[
n(n+1) = (4q + r)(4q + r + 1) = 16q^2 + 4qr + 4q + 4qr + r^2 + r
\]
Notice that \(r^2 + r\) is a multiple of \(4\) only when \(r = 0\) or \(r = 3\).
\end{proof}
